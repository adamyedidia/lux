Imaging scenes that are not directly visible, also called non-line-of-sight (NLoS) imaging, is a difficult and often ill-posed problem. Recently, it has become an area of active study [citations]; methods that rely on visible light to image hidden scenes usually presume that there is something directly visible to both the observer and the hidden scene (see Fig.~\ref{fig:doorway} for an illustration of such a scenario). In this work, we refer to this visible area as the \emph{observation plane.} 

Past methods that rely on human-visible light to image hidden scenes can be divided into one of two categories: \emph{active} methods, which introduce light into the scene and make use of known or measured properties of the introduced light, such as time of return, to image the hidden scene [citations]. \emph{Passive} methods, on the other hand, rely exclusively on ambient light from the scene, such as secondary reflections on the observation plane, to infer the contents of the hidden scene [citations]. In general, active methods have are more powerful and have a wider variety of tools available with which to do imaging, but passive methods are more widely applicable, since they can be deployed even without the help of lasers or time-of-flight cameras. However, passive methods do usually rely on the hidden scene being ambiently lit and cannot be used to image dark scenes. The methods presented in this paper are passive methods, and as such assume the hidden scene to be lit.

Passive methods suffer from the fact that in real-world settings, only a two-dimensional array of observations can be observed (see e.g. Fig.~\ref{fig:doorway}), but the scene producing those observations is three-dimensional. Hence, the problem in such cases is inherently ill-posed. Past methods have resolved this issue by either assuming the scene lies on a lower-dimensional manifold, thereby only reconstructing only a lower-dimensional projection of the scene [citations], or making use of a strong spatial prior over realistic scenes to reconstruct [citations]. Our method falls into this former category, as we assume that both the scene and occluder lie on parallel, flat planes. This allows us to model the shadows cast on the observation planes as a simple convolution of these two planes (see Fig~\ref{fig:convolveModel} for an illustration).

Although there has by now been plenty of previous work demonstrating that it is possible to use the presence of an occluder to infer the structure of a hidden scene [citations], this work, to our knowledge, is the first to do so in a \emph{blind} manner, meaning that we know nothing \emph{a priori} about the structure of the occluder. Past work that exploits occlusion either uses scene calibration to get a precise picture of the occluder before system can work [citations] or is limited to situations in which the occluder has some basic, common shape, like a pinhole, pinspeck, or edge [citations]. The blind nature of this problem compounds the already daunting challenge of non-line-of-sight imaging. However, we hope that this will make our method widely applicable in a variety of situations in which occluders are complex but pre-calibration is not an option, such as traffic or search and rescue [citations].


